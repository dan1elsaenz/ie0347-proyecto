% 3_obtencion_modelo_planta.tex

\section{Obtención del modelo de la planta}
Ahora bien, con base en la caracterización realizada en la sección anterior, se procede con la obtención del modelo del circuito eléctrico a controlar, según la Figura \ref{fig:circuito_planta}.

Se tienen dos componentes almacenadores de energía (capacitores), lo cual coincide con la teoría de un modelo de segundo orden.
En el Apéndice \ref{anexoA}, se documentó el procedimiento realizado para obtener la ecuación diferencial de la salida $v_o$ en términos de la entrada $v_s$, así como la función de transferencia del modelo de la planta.

\begin{equation}
    R_1R_2C_1C_2\ddot{v_o} + (R_1C_1+R_1C_2+R_2C_2)\dot{v_o} + v_o = v_s
\end{equation}

Luego, se aplica la transformada de Laplace y se resuelve para $\frac{V_o}{V_s}$.
Se toman condiciones iniciales nulas para el análisis y se obtiene la función de transferencia que modela la planta en cuestión.

\begin{equation}
    H(s)=\frac{1}{R_1 R_2 C_1 C_2 s^2 + (R_1 C_1 + R_1 C_2 +R_2 C_2)s+1}
\end{equation}

Al reemplazar los valores experimentales de los componentes eléctricos de la Figura \ref{fig:circuito_planta}, resulta la función de transferencia que modela la planta:

\begin{equation} \label{funcion_transferencia_planta}
    H(s)=\frac{1}{0.8789s^2 + 2.8162s+1}
\end{equation}

Al analizar la forma de \eqref{funcion_transferencia_planta}, se determina que la función de transferencia que modela la planta efectivamente es de segundo orden.
Además de que es tipo 0, puesto que no contiene polos en el origen.
También, esta representa una respuesta sobreamortiguada, debido a que el valor numérico del factor de amortiguamiento es mayor a 1, específicamente $\xi =1.5020$.

Para el proceso de diseño del controlador, se requiere la ubicación de los polos del modelo de la planta, los cuales se indican en \eqref{polos}.

\begin{equation} \label{polos}
    \begin{gathered}
        s_1= -0.4067 \\
        s_2= -2.7975
    \end{gathered}
\end{equation}

A modo de verificación gráfica, se simuló la respuesta temporal del sistema de lazo abierto a un escalón unitario en la Figura \ref{fig:respuesta-temporal-lazo-abierto}, a partir de \eqref{funcion_transferencia_planta}.
En esta, se evidencia el comportamiento sobreamortiguado de la planta, debido a la ausencia de un sobrepaso respecto a la entrada.

\begin{figure}[htbp]
    \centering
    \resizebox{0.95\columnwidth}{!}{
    \def\svgwidth{1\columnwidth}
    \import{images/respuesta_lazo_abierto}{respuesta_lazo_abierto.pdf_tex}
    }
    \caption{Simulación de la respuesta temporal del modelo de la planta en lazo abierto.}
    \label{fig:respuesta-temporal-lazo-abierto}
\end{figure}

Asimismo, en la Figura \ref{fig:bode-lazo-abierto}, se observa la respuesta en frecuencia del modelo de la planta, donde destaca la pendiente de \SI{-40}{\deci\bel}/década característica de un sistema de segundo grado.

\begin{figure}[htbp]
    \centering
    \resizebox{0.95\columnwidth}{!}{
    \def\svgwidth{1.2\columnwidth}
    \import{images/bode_lazo_abierto}{bode_lazo_abierto.pdf_tex}
    }
    \caption{Diagrama de Bode de lazo abierto del sistema.}
    \label{fig:bode-lazo-abierto}
\end{figure}
