% 4_diseno_controlador.tex

\section{Diseño del controlador}

Para el diseño del controlador $C$ del sistema de control realimentado diseñado, se debieron analizar las especificaciones dadas en el enunciado respecto a cambios tipo escalón unitario en la referencia:
\begin{itemize}
    \item Error estacionario igual a cero.
    \item Tiempo de asentamiento al $2\%$ menor o igual a \SI{5}{\second}.
    \item Sobrepaso máximo menor al $5\%$.
    \item Margen de fase mayor a $30^\circ$ y de ganancia superior a \SI{6}{\deci\bel}.
\end{itemize}

Con base en dichas consideraciones, se diseñó el controlador utilizando la técnica del \textit{Lugar Geométrico de las Raíces} (LGR).
Inicialmente, por la naturaleza del sistema, tipo 0, se requiere colocar un integrador en la función de transferencia del controlador para conseguir un error estacionario nulo.
Sin embargo, bajo esta configuración, se tiene un tiempo de asentamiento al $2\%$ de \SI{27.9}{\second}.

Luego, se aplicó la teoría de cancelación de polos del modelo de la planta, por medio de la adición de ceros en el controlador, tal que sus ubicaciones coincidan.
Esto se realiza de forma que se cancela el polo dominante, según \eqref{polos}, $s = -0.4067$.
Así, al agregar un cero en la ubicación anterior, se elimina efectivamente la constante de tiempo asociada a este polo y el tiempo de respuesta del sistema disminuye.\cite{b1}

En la Figura \ref{fig:root-locus}, se muestra la ubicación de los polos seleccionados para el controlador.
Se colocaron requerimientos de diseño para ajustar el diseño del controlador a las especificaciones dadas al inicio.

\begin{figure}[htbp]
    \centering
    \resizebox{0.95\columnwidth}{!}{
    \def\svgwidth{\columnwidth}
    \import{images/root_locus}{root_locus.pdf_tex}
    }
    \caption{Diseño del controlador utilizando la técnica de LGR.}
    \label{fig:root-locus}
\end{figure}

La línea vertical se coloca de tal forma que se cumpla la especificación de tiempo de asentamiento al $2\%$ menor que \SI{5}{\second}.
Se debe cumplir que $s < -0.8$.
Esto se obtiene a partir de la definición de este requerimiento.

Asimismo, para cumplir con la especificación del sobrepaso máximo, se tiene que esta variable depende de $\xi$.
A partir de la definición, se obtiene que se tiene un ángulo respecto al eje horizontal límite de $\theta = 46.3619^\circ$ para cumplir con esta especificación.
En el Apéndice \ref{anexoB}, se muestran los cálculos detallados de estos valores numéricos según la teoría vista en el curso.

Ahora bien, para cumplir con ambos parámetros, se requiere que la ubicación de los polos esté en la región blanca, lo cual se cumple en el caso del diseño.

Con base en el análisis descrito anteriormente, se determinó el controlador resultante.

\begin{equation} \label{eq:controlador_continuo}
    C(s) = \frac{1.5996 \left(s+0.4060\right)}{s}
\end{equation}

\section{Simulaciones en tiempo continuo y discreto}

Para la verificación de los resultados correspondientes al diseño del controlador, se realizaron las simulaciones en tiempo continuo y discreto mediante el uso de las herramientas \texttt{Sisotool} y \texttt{Simulink} de MATLAB.

\subsection{Tiempo continuo}

Primero, en la Figura \ref{fig:respuesta-temporal-continua}, se muestra el resultado de la simulación del controlador continuo exportado de \texttt{Sisotool} con la planta en cuestión y el lazo de realimentación.
Esta simulación está asociada al diagrama de bloques de la Figura \ref{fig:diagrama-bloques-lazo-cerrado}.

\begin{figure}[htbp]
    \centering
    \resizebox{0.95\columnwidth}{!}{
    \def\svgwidth{1\columnwidth}
    \import{images/respuesta_continua}{respuesta_continua.pdf_tex}
    }
    \caption{Simulación de la respuesta temporal en lazo cerrado con el modelo del controlador y de la planta en tiempo continuo.}
    \label{fig:respuesta-temporal-continua}
\end{figure}

Según el diseño realizado, se tiene una respuesta sobreamortiguada, lo cual indica que no existe sobrepaso respecto a la señal de referencia del escalón unitario.
Asimismo, se obtuvo un tiempo de asentamiento al $2\%$ de \SI{4.64}{\second}, lo cual cumple con la especificación proporcionada.
Como se agregó un integrador al controlador, se determinó que efectivamente, el error estacionario es de $0 \%$.

Además, la señal de control $u$ en respuesta al escalón unitario posee una amplitud máxima menor a \SI{2}{\volt} (\SI{1.6906}{\volt} específicamente).
Este corresponde a un valor aceptable respecto a lo que puede proporcionar el controlador utilizado, entre 0 y \SI{5}{\volt}.
También, se observa que conforme la salida se aproxima a la referencia, la magnitud de la señal de control disminuye, lo cual es esperable según la teoría y tiende a $1$.

Ahora bien, aparte del análisis en el dominio temporal, se requiere estudiar el comportamiento en el dominio de la frecuencia, para el cual interesan dos cantidades: el margen de ganancia y de fase.

Según la Figura \ref{fig:bode-lazo-cerrado}, se determinó en MATLAB un margen de ganancia infinito (no se da el cruce en la fase de $-180^\circ$).
Esto implica que es posible aumentar la ganancia indefinidamente sin llegar a una inestabilidad según este criterio.

Respecto al margen de fase, se reportó un valor de $77.2^\circ$.
Típicamente, se busca que el margen de fase se encuentre entre $30^\circ$ y $60^\circ$; sin embargo, este resultado numérico implica una respuesta estable, pero más lenta respecto a un margen de fase menor, lo cual no es un inconveniente respecto a las especificaciones conseguidas con la respuesta temporal.

\begin{figure}[htbp]
    \centering
    \resizebox{0.95\columnwidth}{!}{
    \def\svgwidth{1.2\columnwidth}
    \import{images/bode_lazo_cerrado}{bode_lazo_cerrado.pdf_tex}
    }
    \caption{Diagrama de Bode del sistema en lazo cerrado.}
    \label{fig:bode-lazo-cerrado}
\end{figure}

\subsection{Tiempo discreto}

Para la implementación del sistema de control realimentado en Arduino, es necesario convertir el controlador diseñado a una forma discreta, debido a la naturaleza digital de los microcontroladores.
Para ello, se discretizó el controlador continuo diseñado de \eqref{eq:controlador_continuo} con el comando \texttt{cd2} de MATLAB, específicamente con el método \texttt{zoh} correspondiente a \textit{zero-order hold}.

Respecto al tiempo de muestreo seleccionado, se observa que el pin digital 6 PWM permite muestrear a una frecuencia de \SI{980}{\hertz}, lo cual equivale a un tiempo de \SI{1.0204}{\milli\second}.
Asimismo, como se deben medir los valores de tensión eléctrica de referencia, de control y de salida al menos 2 veces por segundo, según el enunciado, esto implica un límite superior de \SI{500}{\milli\second}.
Ahora bien, con los límites establecidos anteriormente, hay que considerar que a menores tiempos de muestreo, es más probable que se presente inestabilidad al discretizar el controlador y probarlo experimentalmente.
Por lo tanto, por las constantes de tiempo asociadas a la planta (cercanas a \SI{1}{\second}) y para evitar tener \textit{aliasing} significativo, se utilizó un tiempo de muestreo de \SI{100}{\milli\second}.\cite{b2}

El resultado del proceso de discretización resulta en la función de transferencia para el controlador en el dominio $z$:

\begin{equation} \label{eq:controlador_discreto}
    C(z) = \frac{1.5996 \left( z - 0.9594 \right)}{z-1}
\end{equation}

A partir de esta expresión, se utilizó la herramienta de MATLAB \texttt{Simulink} para simular la respuesta temporal ante una entrada de escalón unitario.
En la Figura \ref{fig:respuesta-temporal-discreta}, se muestra el resultado obtenido junto con las especificaciones de tiempo de asentamiento al $2\%$, porcentaje de sobrepaso y error estacionario.

\begin{figure}[htbp]
    \centering
    \resizebox{0.95\columnwidth}{!}{
    \def\svgwidth{1\columnwidth}
    \import{images/respuesta_discreta}{respuesta_discreta.pdf_tex}
    }
    \caption{Simulación de la respuesta temporal en lazo cerrado con el modelo del controlador discreto y de la planta continua.}
    \label{fig:respuesta-temporal-discreta}
\end{figure}

Respecto a las especificaciones obtenidas de la simulación, se tuvo un tiempo de asentamiento al $2\%$ de \SI{4.40}{\second}, el cual es menor respecto al caso con el controlador continuo.
Esto se debe a que, cuando se realiza la discretización, la señal de salida presenta un ligero aumento en su valor, lo cual también afecta el porcentaje de sobrepaso respecto a la señal de referencia, pues se midió un $0.0659\%$.
Asimismo, al igual que en el caso continuo, se presenta un error estacionario nulo.

Además, respecto a la señal de control, se observa un aumento poco significativo en su amplitud máxima requerida (\SI{1.6936}{\volt}), lo cual no corresponde a un inconveniente para el controlador.
La señal de salida y de control tardan más tiempo en estabilizarse en el valor de referencia, pero esto ocurre para valores menores a la banda del $2\%$.

En la Figura \ref{fig:respuesta-temporal-discreta} se agregó la señal de salida continua a modo de comparación para poder determinar su parecido.
Con base en los resultados obtenidos para el caso discreto respecto al continuo, se determinó que el proceso de discretización cumple con los parámetros requeridos y se asemeja a la señal de salida en tiempo continuo.

