% 3_caracterizacion_sistema.tex

\section{Caracterización del sistema}

En primer lugar, es necesario describir el sistema lineal invariante con el tiempo para el cual se diseñó el sistema de control realimentado.
La planta consiste en un circuito RC de segundo orden sobreamortiguado, tipo 0, compuesto por dos capacitores y dos resistores.
La planta junto con los valores medidos experimentalmente se muestran en la Figura \ref{fig:circuito_planta}.
Se requería controlar la tensión eléctrica de salida $v_o$, a partir de la tensión eléctrica de entrada $v_s$ proporcionada por un Arduino UNO R3.

\begin{figure}[htbp]
    \centering
    \resizebox{0.9\columnwidth}{!}{
    \def\svgwidth{\columnwidth}
    \import{images/circuito_planta}{Circuito_valores.pdf_tex}
    }
    \caption{Circuito eléctrico con los valores medidos en el laboratorio del proceso a controlar.}
    \label{fig:circuito_planta}
\end{figure}

En términos generales, se realizaron dos tipos de análisis: de lazo abierto y de lazo cerrado.
En la Figura \ref{fig:diagrama-bloques-lazo-abierto} se muestra el diagrama de bloques de lazo abierto, donde se señalan el controlador $C$, la planta $P$ y la señal de entrada $u$ y salida $y$.
En este caso, el controlador corresponde al Arduino UNO R3 con un microcontrolador ATmega328P y la planta consiste en el circuito eléctrico mencionado anteriormente.

\begin{figure}[htbp]
    \centering
    \resizebox{\columnwidth}{!}{
    \def\svgwidth{1.2\columnwidth}
    \import{images/diagrama_bloques_lazo_abierto}{diagrama_bloques_lazo_abierto.pdf_tex}
    }
    \caption{Diagrama de bloques del sistema en lazo abierto.}
    \label{fig:diagrama-bloques-lazo-abierto}
\end{figure}

Ahora bien, para el sistema de control realimentado diseñado, se debe considerar el sistema con el lazo de control cerrado, el cual se muestra en la Figura \ref{fig:diagrama-bloques-lazo-cerrado}.

\begin{figure}[htbp]
    \centering
    \resizebox{\columnwidth}{!}{
    \def\svgwidth{1.2\columnwidth}
    \import{images/diagrama_bloques_lazo_cerrado}{diagrama_bloques_lazo_cerrado.pdf_tex}
    }
    \caption{Diagrama de bloques del sistema en lazo cerrado.}
    \label{fig:diagrama-bloques-lazo-cerrado}
\end{figure}

En este, se muestran las señales de interés en el proceso de diseño y operación: la referencia $r$ (valor objetivo de tensión eléctrica $v_o$), la señal de control $u$ (salida del Arduino del pin digital 6 PWM, que corresponde a la tensión de entrada de la planta) y la salida $y$ (tensión eléctrica $v_o$).
Asimismo, se muestran los componentes de interés correspondientes al controlador y la planta.

Además, se considera que el sistema posee dos interfaces: el convertidor A/D en los pines analógicos, que consiste en el sensor de tensión eléctrica del Arduino; y un convertidor D/A, el cual se desempeña como el actuador del sistema que envía la señal de control de tensión eléctrica a la planta, a través de los pines digitales del Arduino.

